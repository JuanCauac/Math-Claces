\documentclass[a4paper,12pt]{article}
\usepackage[utf8x]{inputenc}
\usepackage{amssymb}
\usepackage{xlop}
\usepackage{xcolor}
\input{longdiv}
\usepackage{graphicx}
\usepackage{wrapfig}
\usepackage{float}
%PDF metadata
\usepackage[pdftex, 
pdfauthor={Juan Carlos Alvarez},
pdfsubject={matematicas, suma de fracciones},
pdfkeywords={negativos, suma, resta, fracciones}]{hyperref}
%opening
\hypersetup{pdftitle={Solucion tarea suma de fracciones}}
\title{Solución}
\author{}
\date{}
\begin{document}
\maketitle
 Suma  y resta las fracciones\vspace{1cm}\\ 
1) $\frac{3}{8}+\frac{7}{8}=\frac{10}{8}$\vspace{1cm}\\ 
2) $\frac{7}{2}+\frac{8}{2}=\frac{15}{2}$\vspace{1cm}\\ 
3) $\frac{2}{5}+\frac{2}{6}+\frac{4}{9}  =\frac{36+30+40}{90}= \frac{106}{90}  $\vspace{1cm}\\ 
4) $\frac{2}{5}+\frac{7}{2}+\frac{5}{3}  =\frac{12+105+50}{30}= \frac{167}{30}  $\vspace{1cm}\\ 
5) $\frac{10}{4}+\frac{7}{3}+\frac{6}{11}  =\frac{330+308+72}{132}= \frac{710}{132}  $\vspace{1cm}\\ 

 \pagebreak 
 Suma  y resta las fracciones\vspace{1cm}\\ 
1) $\frac{10}{6}+\frac{7}{6}=\frac{17}{6}$\vspace{1cm}\\ 
2) $\frac{7}{9}+\frac{6}{9}=\frac{13}{9}$\vspace{1cm}\\ 
3) $\frac{1}{2}+\frac{7}{9}+\frac{9}{9}  =\frac{9+14+18}{18}= \frac{41}{18}  $\vspace{1cm}\\ 
4) $\frac{8}{10}+\frac{5}{2}+\frac{8}{7}  =\frac{56+175+80}{70}= \frac{311}{70}  $\vspace{1cm}\\ 
5) $\frac{10}{4}+\frac{9}{3}+\frac{10}{11}  =\frac{330+396+120}{132}= \frac{846}{132}  $\vspace{1cm}\\ 

 \pagebreak 
 Suma  y resta las fracciones\vspace{1cm}\\ 
1) $\frac{9}{8}+\frac{5}{8}=\frac{14}{8}$\vspace{1cm}\\ 
2) $\frac{7}{11}+\frac{8}{11}=\frac{15}{11}$\vspace{1cm}\\ 
3) $\frac{6}{9}+\frac{8}{11}+\frac{8}{5}  =\frac{330+360+792}{495}= \frac{1482}{495}  $\vspace{1cm}\\ 
4) $\frac{5}{10}+\frac{8}{4}+\frac{4}{2}  =\frac{10+40+40}{20}= \frac{90}{20}  $\vspace{1cm}\\ 
5) $\frac{1}{11}+\frac{7}{10}+\frac{7}{11}  =\frac{10+77+70}{110}= \frac{157}{110}  $\vspace{1cm}\\ 

 \pagebreak 
 Suma  y resta las fracciones\vspace{1cm}\\ 
1) $\frac{1}{3}+\frac{2}{3}=\frac{3}{3}$\vspace{1cm}\\ 
2) $\frac{5}{2}+\frac{7}{2}=\frac{12}{2}$\vspace{1cm}\\ 
3) $\frac{5}{3}+\frac{7}{4}+\frac{8}{5}  =\frac{100+105+96}{60}= \frac{301}{60}  $\vspace{1cm}\\ 
4) $\frac{7}{5}+\frac{9}{2}+\frac{9}{3}  =\frac{42+135+90}{30}= \frac{267}{30}  $\vspace{1cm}\\ 
5) $\frac{7}{3}+\frac{3}{8}+\frac{5}{9}  =\frac{168+27+40}{72}= \frac{235}{72}  $\vspace{1cm}\\ 

 \pagebreak 
 Suma  y resta las fracciones\vspace{1cm}\\ 
1) $\frac{8}{3}+\frac{1}{3}=\frac{9}{3}$\vspace{1cm}\\ 
2) $\frac{4}{10}+\frac{9}{10}=\frac{13}{10}$\vspace{1cm}\\ 
3) $\frac{4}{5}+\frac{7}{10}+\frac{7}{3}  =\frac{24+21+70}{30}= \frac{115}{30}  $\vspace{1cm}\\ 
4) $\frac{5}{7}+\frac{5}{7}+\frac{9}{8}  =\frac{40+40+63}{56}= \frac{143}{56}  $\vspace{1cm}\\ 
5) $\frac{7}{3}+\frac{10}{5}+\frac{1}{4}  =\frac{140+120+15}{60}= \frac{275}{60}  $\vspace{1cm}\\ 

 \pagebreak 
\end{document}