\documentclass[a4paper,12pt]{article}
\usepackage[utf8x]{inputenc}
\usepackage{amssymb}
\usepackage{xlop}
\usepackage{xcolor}
\input{longdiv}
\usepackage{graphicx}
\usepackage{wrapfig}
\usepackage{float}
%PDF metadata
\usepackage[pdftex, 
pdfauthor={Juan Carlos Alvarez},
pdfsubject={matematicas, negativos y positivos},
pdfkeywords={negativos, positivos, sumas}]{hyperref}
%opening
\hypersetup{pdftitle={Solucion tarea negativos y positivos}}
\title{Solución}
\author{}
\date{}
\begin{document}
\maketitle
 Expreza como multiplicación y  resuelve.
 Ejemplo: $2^3=2 \times 2 \times 2 = 8$  \vspace{0.5cm}\\ 
1) $5^2=5\times  5=25$ \vspace{1cm}\\ 
2) $2^0= 1$ \vspace{1cm}\\ 
3) $9^2=9\times  9=81$ \vspace{1cm}\\ 
4) $6^1= 6=6$ \vspace{1cm}\\ 
5) $9^3=9\times 9\times  9=729$ \vspace{1cm}\\ 
 Expreza como exponentes.
 Ejemplo: $\sqrt[3]{5^2}=5^\frac{5}{3} $  \vspace{0.5cm}\\ 
6) $\sqrt[1]{3}^1=3^\frac{1}{1 }$\vspace{1cm}\\ 
7) $\sqrt[4]{0}^-1=0^\frac{-1}{4 }$\vspace{1cm}\\ 
8) $\sqrt[4]{8}^1=8^\frac{1}{4 }$\vspace{1cm}\\ 
9) $\sqrt[3]{4}^3=4^\frac{3}{3 }$\vspace{1cm}\\ 
10) $\sqrt[4]{5}^3=5^\frac{3}{4 }$\vspace{1cm}\\ 

 \pagebreak 
1) $7^3=7\times 7\times  7=343$ \vspace{1cm}\\ 
2) $5^0= 1$ \vspace{1cm}\\ 
3) $1^4=1\times 1\times 1\times  1=1$ \vspace{1cm}\\ 
4) $6^4=6\times 6\times 6\times  6=1296$ \vspace{1cm}\\ 
5) $2^3=2\times 2\times  2=8$ \vspace{1cm}\\ 
 Expreza como exponentes.
 Ejemplo: $\sqrt[3]{5^2}=5^\frac{5}{3} $  \vspace{0.5cm}\\ 
6) $\sqrt[3]{0}^-3=0^\frac{-3}{3 }$\vspace{1cm}\\ 
7) $\sqrt[5]{9}^-2=9^\frac{-2}{5 }$\vspace{1cm}\\ 
8) $\sqrt[2]{5}^-1=5^\frac{-1}{2 }$\vspace{1cm}\\ 
9) $\sqrt[2]{1}^1=1^\frac{1}{2 }$\vspace{1cm}\\ 
10) $\sqrt[1]{1}^-2=1^\frac{-2}{1 }$\vspace{1cm}\\ 

 \pagebreak 
1) $5^0= 1$ \vspace{1cm}\\ 
2) $1^1= 1=1$ \vspace{1cm}\\ 
3) $5^3=5\times 5\times  5=125$ \vspace{1cm}\\ 
4) $9^0= 1$ \vspace{1cm}\\ 
5) $1^0= 1$ \vspace{1cm}\\ 
 Expreza como exponentes.
 Ejemplo: $\sqrt[3]{5^2}=5^\frac{5}{3} $  \vspace{0.5cm}\\ 
6) $\sqrt[2]{8}^1=8^\frac{1}{2 }$\vspace{1cm}\\ 
7) $\sqrt[2]{7}^1=7^\frac{1}{2 }$\vspace{1cm}\\ 
8) $\sqrt[2]{7}^4=7^\frac{4}{2 }$\vspace{1cm}\\ 
9) $\sqrt[3]{1}^-3=1^\frac{-3}{3 }$\vspace{1cm}\\ 
10) $\sqrt[5]{6}^0=6^\frac{0}{5 }$\vspace{1cm}\\ 

 \pagebreak 
1) $2^4=2\times 2\times 2\times  2=16$ \vspace{1cm}\\ 
2) $4^1= 4=4$ \vspace{1cm}\\ 
3) $2^0= 1$ \vspace{1cm}\\ 
4) $4^3=4\times 4\times  4=64$ \vspace{1cm}\\ 
5) $1^4=1\times 1\times 1\times  1=1$ \vspace{1cm}\\ 
 Expreza como exponentes.
 Ejemplo: $\sqrt[3]{5^2}=5^\frac{5}{3} $  \vspace{0.5cm}\\ 
6) $\sqrt[3]{8}^4=8^\frac{4}{3 }$\vspace{1cm}\\ 
7) $\sqrt[1]{0}^-5=0^\frac{-5}{1 }$\vspace{1cm}\\ 
8) $\sqrt[4]{4}^0=4^\frac{0}{4 }$\vspace{1cm}\\ 
9) $\sqrt[2]{6}^-5=6^\frac{-5}{2 }$\vspace{1cm}\\ 
10) $\sqrt[5]{5}^-5=5^\frac{-5}{5 }$\vspace{1cm}\\ 

 \pagebreak 
1) $5^4=5\times 5\times 5\times  5=625$ \vspace{1cm}\\ 
2) $5^2=5\times  5=25$ \vspace{1cm}\\ 
3) $8^0= 1$ \vspace{1cm}\\ 
4) $7^2=7\times  7=49$ \vspace{1cm}\\ 
5) $2^3=2\times 2\times  2=8$ \vspace{1cm}\\ 
 Expreza como exponentes.
 Ejemplo: $\sqrt[3]{5^2}=5^\frac{5}{3} $  \vspace{0.5cm}\\ 
6) $\sqrt[4]{0}^-4=0^\frac{-4}{4 }$\vspace{1cm}\\ 
7) $\sqrt[1]{8}^3=8^\frac{3}{1 }$\vspace{1cm}\\ 
8) $\sqrt[5]{8}^1=8^\frac{1}{5 }$\vspace{1cm}\\ 
9) $\sqrt[2]{4}^-1=4^\frac{-1}{2 }$\vspace{1cm}\\ 
10) $\sqrt[1]{4}^1=4^\frac{1}{1 }$\vspace{1cm}\\ 

 \pagebreak 
\end{document}