\documentclass[a4paper,12pt]{article}
\usepackage[utf8x]{inputenc}
\usepackage{amssymb}
\usepackage{xlop}
\usepackage{xcolor}
\input{longdiv}
\usepackage{graphicx}
\usepackage{wrapfig}
\usepackage{float}
%PDF metadata
\usepackage[pdftex, 
pdfauthor={Juan Carlos Alvarez},
pdfsubject={matematicas, negativos y positivos},
pdfkeywords={negativos, positivos, sumas}]{hyperref}
%opening
\hypersetup{pdftitle={Tarea negativos y positivos}}
\title{Tarea}
\author{}
\date{}
\begin{document}
\maketitle
 Expreza como multiplicación y  resuelve.
 Ejemplo: $2^3=2 \times 2 \times 2 = 8$  \vspace{0.5cm}\\ 
1) $5^2=$\vspace{1cm}\\ 
2) $2^0=$\vspace{1cm}\\ 
3) $9^2=$\vspace{1cm}\\ 
4) $6^1=$\vspace{1cm}\\ 
5) $9^3=$\vspace{1cm}\\ 
 Expreza como exponentes.
 Ejemplo: $\sqrt[3]{5^2}=5^\frac{5}{3} $  \vspace{0.5cm}\\ 
6) $\sqrt[1]{3}^1=$\vspace{1cm}\\ 
7) $\sqrt[4]{0}^-1=$\vspace{1cm}\\ 
8) $\sqrt[4]{8}^1=$\vspace{1cm}\\ 
9) $\sqrt[3]{4}^3=$\vspace{1cm}\\ 
10) $\sqrt[4]{5}^3=$\vspace{1cm}\\ 

 \pagebreak 
1) $7^3=$\vspace{1cm}\\ 
2) $5^0=$\vspace{1cm}\\ 
3) $1^4=$\vspace{1cm}\\ 
4) $6^4=$\vspace{1cm}\\ 
5) $2^3=$\vspace{1cm}\\ 
 Expreza como exponentes.
 Ejemplo: $\sqrt[3]{5^2}=5^\frac{5}{3} $  \vspace{0.5cm}\\ 
6) $\sqrt[3]{0}^-3=$\vspace{1cm}\\ 
7) $\sqrt[5]{9}^-2=$\vspace{1cm}\\ 
8) $\sqrt[2]{5}^-1=$\vspace{1cm}\\ 
9) $\sqrt[2]{1}^1=$\vspace{1cm}\\ 
10) $\sqrt[1]{1}^-2=$\vspace{1cm}\\ 

 \pagebreak 
1) $5^0=$\vspace{1cm}\\ 
2) $1^1=$\vspace{1cm}\\ 
3) $5^3=$\vspace{1cm}\\ 
4) $9^0=$\vspace{1cm}\\ 
5) $1^0=$\vspace{1cm}\\ 
 Expreza como exponentes.
 Ejemplo: $\sqrt[3]{5^2}=5^\frac{5}{3} $  \vspace{0.5cm}\\ 
6) $\sqrt[2]{8}^1=$\vspace{1cm}\\ 
7) $\sqrt[2]{7}^1=$\vspace{1cm}\\ 
8) $\sqrt[2]{7}^4=$\vspace{1cm}\\ 
9) $\sqrt[3]{1}^-3=$\vspace{1cm}\\ 
10) $\sqrt[5]{6}^0=$\vspace{1cm}\\ 

 \pagebreak 
1) $2^4=$\vspace{1cm}\\ 
2) $4^1=$\vspace{1cm}\\ 
3) $2^0=$\vspace{1cm}\\ 
4) $4^3=$\vspace{1cm}\\ 
5) $1^4=$\vspace{1cm}\\ 
 Expreza como exponentes.
 Ejemplo: $\sqrt[3]{5^2}=5^\frac{5}{3} $  \vspace{0.5cm}\\ 
6) $\sqrt[3]{8}^4=$\vspace{1cm}\\ 
7) $\sqrt[1]{0}^-5=$\vspace{1cm}\\ 
8) $\sqrt[4]{4}^0=$\vspace{1cm}\\ 
9) $\sqrt[2]{6}^-5=$\vspace{1cm}\\ 
10) $\sqrt[5]{5}^-5=$\vspace{1cm}\\ 

 \pagebreak 
1) $5^4=$\vspace{1cm}\\ 
2) $5^2=$\vspace{1cm}\\ 
3) $8^0=$\vspace{1cm}\\ 
4) $7^2=$\vspace{1cm}\\ 
5) $2^3=$\vspace{1cm}\\ 
 Expreza como exponentes.
 Ejemplo: $\sqrt[3]{5^2}=5^\frac{5}{3} $  \vspace{0.5cm}\\ 
6) $\sqrt[4]{0}^-4=$\vspace{1cm}\\ 
7) $\sqrt[1]{8}^3=$\vspace{1cm}\\ 
8) $\sqrt[5]{8}^1=$\vspace{1cm}\\ 
9) $\sqrt[2]{4}^-1=$\vspace{1cm}\\ 
10) $\sqrt[1]{4}^1=$\vspace{1cm}\\ 

 \pagebreak 
\end{document}