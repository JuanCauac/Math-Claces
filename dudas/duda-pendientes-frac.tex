\documentclass[a4paper,10pt]{article}
\usepackage[utf8x]{inputenc}
\usepackage{amssymb}
\usepackage{xlop}
\usepackage{xcolor}
\input{longdiv}
\usepackage{graphicx}
\usepackage{wrapfig}
\usepackage{float}

%opening
\title{}
\author{}
\date{}
\begin{document}



Por ejemplo:
 $A= (-2, 3)$ y $B=(5, -2)$

Recordamos la formula para la pendiente:
$$m=\frac{y_1-y_0}{x_1-x_0}$$

Substituimos:
$m=\frac{3-(-2)}{-2-5}=\frac{5}{-7}=\mathbf{-\frac{5}{7}}$

Recordamos la formula para la ordenada al origen:
$$b=y_0-mx_0$$

Substutuimos:
$b=-2-(-\frac{5}{7})5=-2-(-\frac{5}{7})\frac{5}{1}=-2-(-\frac{25}{7})=-2+\frac{25}{7}=-\frac{2}{1}+\frac{25}{7}=$
\\
$-\frac{14}{7}+\frac{25}{7}=\mathbf{\frac{11}{7}}$



\end{document}

